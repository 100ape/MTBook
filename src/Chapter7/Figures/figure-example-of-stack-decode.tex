%%%------------------------------------------------------------------------------------------------------------
%%% 栈解码
\begin{center}
\begin{tikzpicture}
\begin{scope}
{
\node [anchor=north,inner sep=2pt,fill=red!40,minimum height=2em,minimum width=3em] (h0) at (0,0) {\scriptsize{null}};
\node [anchor=north west,inner sep=1.5pt,fill=black] (hl0) at (h0.north west) {\scriptsize{{\color{white} \textbf{0}}}};
\node [anchor=north,rotate=90,inner sep=1pt,minimum width=2em,fill=black] (pt0) at (h0.east) {\tiny{{\color{white} \textbf{$\funp{P}$=1}}}};
}
{
\node [anchor=west,inner sep=2pt,fill=red!40,minimum height=2em,minimum width=3em] (h13) at ([xshift=2.1em,yshift=6em]h0.east) {\scriptsize{there is}};
\node [anchor=west,inner sep=2pt,minimum height=2em,minimum width=3em] (h12) at ([xshift=2.1em,yshift=3.5em]h0.east) {\small{\textbf{...}}};
\node [anchor=west,inner sep=2pt,fill=red!40,minimum height=2em,minimum width=3em] (h1) at ([xshift=2.1em]h0.east) {\scriptsize{tabel}};

\node [anchor=north west,inner sep=1.0pt,fill=black] (hl1) at (h1.north west) {\scriptsize{{\color{white} \textbf{1}}}};
\node [anchor=north west,inner sep=1.0pt,fill=black] (hl3) at (h13.north west) {\scriptsize{{\color{white} \textbf{3}}}};


\node [anchor=north,rotate=90,inner sep=1pt,minimum width=2em,fill=black] (pt1) at (h1.east) {\tiny{{\color{white} \textbf{$\funp{P}$=0.2}}}};
\node [anchor=north,rotate=90,inner sep=1pt,minimum width=2em,fill=black] (pt3) at (h13.east) {\tiny{{\color{white} \textbf{$\funp{P}$=0.5}}}};

\node [anchor=west,inner sep=2pt,fill=red!40,minimum height=2em,minimum width=3em] (h2) at ([xshift=2.1em]h1.east) {\scriptsize{have}};
\node [anchor=west,inner sep=2pt,minimum height=2em,minimum width=3em] (h22) at ([xshift=2.1em]h12.east) {\small{\textbf{...}}};
\node [anchor=west,inner sep=2pt,fill=red!40,minimum height=2em,minimum width=3em] (h23) at ([xshift=2.1em]h13.east) {\scriptsize{an}};
\node [anchor=west,inner sep=2pt,fill=red!40,minimum height=2em,minimum width=3em] (h3) at ([xshift=2.1em]h2.east) {\scriptsize{there is}};
\node [anchor=west,inner sep=2pt,minimum height=2em,minimum width=3em] (h32) at ([xshift=2.1em]h22.east) {\small{\textbf{...}}};
\node [anchor=west,inner sep=2pt,fill=red!40,minimum height=2em,minimum width=3em] (h33) at ([xshift=2.1em]h23.east) {\scriptsize{an apple}};

\node [anchor=north west,inner sep=1.0pt,fill=black] (hl2) at (h2.north west) {\scriptsize{{\color{white} \textbf{3}}}};
\node [anchor=north west,inner sep=1.0pt,fill=black] (hl23) at (h23.north west) {\scriptsize{{\color{white} \textbf{4}}}};
\node [anchor=north west,inner sep=1.0pt,fill=black] (hl3) at (h3.north west) {\scriptsize{{\color{white} \textbf{2}}}};
\node [anchor=north west,inner sep=1.0pt,fill=black] (hl33) at (h33.north west) {\scriptsize{{\color{white} \textbf{4-5}}}};

\node [anchor=north,rotate=90,inner sep=1pt,minimum width=2em,fill=black] (pt2) at (h2.east) {\tiny{{\color{white} \textbf{$\funp{P}$=0.5}}}};
\node [anchor=north,rotate=90,inner sep=1pt,minimum width=2em,fill=black] (pt23) at (h23.east) {\tiny{{\color{white} \textbf{$\funp{P}$=0.5}}}};
\node [anchor=north,rotate=90,inner sep=1pt,minimum width=2em,fill=black] (pt3) at (h3.east) {\tiny{{\color{white} \textbf{$\funp{P}$=0.5}}}};
\node [anchor=north,rotate=90,inner sep=1pt,minimum width=2em,fill=black] (pt33) at (h33.east) {\tiny{{\color{white} \textbf{$\funp{P}$=0.5}}}};
}
\node [anchor=north] (l0) at ([xshift=0.2em,yshift=-0.7em]h0.south) {\small{\textbf{未译词}}};
\node [anchor=north] (l1) at ([xshift=0.3em,yshift=-0.7em]h1.south) {\small{\textbf{已译}1\textbf{词}}};
\node [anchor=north] (l2) at ([xshift=0.3em,yshift=-0.7em]h2.south) {\small{\textbf{已译}2\textbf{词}}};
\node [anchor=north] (l3) at ([xshift=0.3em,yshift=-0.7em]h3.south) {\small{\textbf{已译}3\textbf{词}}};

\begin{pgfonlayer}{background}
\node [rectangle,inner sep=0.3em,fill=blue!10] [fit = (h0) (pt0)] (box0) {};
\node [rectangle,inner sep=0.3em,fill=blue!10] [fit = (h1) (pt1) (h13)] (box1) {};
\node [rectangle,inner sep=0.3em,fill=blue!10] [fit = (h2) (pt2) (h23)] (box2) {};
\node [rectangle,inner sep=0.3em,fill=blue!10] [fit = (h3) (pt3) (h33)] (box3) {};
\end{pgfonlayer}

{
\draw [->,thick,red] (h13.north).. controls +(60:0.5) and +(120:0.5) .. (h23.north);
\draw [->,thick,red] (h13.north).. controls +(58:0.8) and +(122:0.8) .. (h33.north);
\draw [->,thick,red] (h1.north).. controls +(60:0.5) and +(120:0.5) .. (h2.north);
\draw [->,thick,red] (h2.north).. controls +(60:0.5) and +(120:0.5) .. (h3.north);
}
\node [anchor=south east] (wtranslabel) at ([xshift=-2.4em,yshift=-2.26em]h0.south west) {\small{\textbf{假设堆栈}}};
\node [anchor=east,inner sep=2pt,fill=blue!10,minimum height=1em,minimum width=2em] (stacklabel) at ([xshift=-0.1em]wtranslabel.west) {};
{
\node [anchor=east] (line1) at ([xshift=-1.0em,yshift=0.45em]h0.west) {\small{0号栈包含空假设}};
}
{
\node [anchor=east] (line2) at ([xshift=-2.3em,yshift=0.44em]h13.west) {\small{通过假设扩展产生新的假设}};
\node [anchor=north west] (line3) at ([yshift=0.1em]line2.south west) {\small{并不断地被存入假设堆栈中}};
}
\begin{pgfonlayer}{background}
{
\node [rectangle,inner sep=0.1em,fill=green!10,draw,thick,rounded corners=0.3em] [fit = (line1)] (box1) {};
}
{
\node [rectangle,inner sep=0.1em,fill=red!10,draw,thick,rounded corners=0.3em] [fit = (line2) (line3)] (box2) {};
}
\end{pgfonlayer}

\end{scope}
\end{tikzpicture}
\end{center}