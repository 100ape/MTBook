
	
\begin{tikzpicture}
\begin{scope}
{\small
\node [anchor=north,rectangle,draw, inner sep=0mm,minimum height=3em,minimum width=6em,rounded corners=5pt,thick,fill=blue!10!white] (n1) at (0, 0) {数据处理};

\node [anchor=west,rectangle,draw, inner sep=0mm,minimum height=3em,minimum width=6em,rounded corners=5pt,thick,fill=yellow!10!white] (n2) at ([xshift=3em,yshift=0em]n1.east) {训练};

\node [anchor=south,rectangle,draw, inner sep=0mm,minimum height=3em,minimum width=6em,rounded corners=5pt,thick,fill=red!10!white] (n3) at ([xshift=0em,yshift=2em]n2.north) {架构设计};

\node [anchor=north,rectangle,draw, inner sep=0mm,minimum height=3em,minimum width=6em,rounded corners=5pt,thick,fill=green!10!white] (n4) at ([xshift=0em,yshift=-2em]n2.south) {推断};
}
\draw [-,very thick] ([xshift=0em,yshift=0em]n1.south)  -- ([xshift=0em,yshift=-3.25em]n1.south);
\draw [->,very thick] ([xshift=-5.5em,yshift=0em]n4.west) --  ([xshift=0em,yshift=0em]n4.west);

\draw [->,very thick] ([xshift=0em,yshift=0em]n1.east) -- ([xshift=0em,yshift=0em]n2.west);
\draw [->,very thick] ([xshift=0em,yshift=0em]n3.south) -- ([xshift=0em,yshift=0em]n2.north);
\draw [->,very thick] ([xshift=0em,yshift=0em]n2.south) -- ([xshift=0em,yshift=0em]n4.north);

{\footnotesize
\node [anchor=west] (n11) at ([xshift=-13em,yshift=2em]n1.west) {对训练和测试数据进行};
\node [anchor=west] (n12) at ([xshift=0em,yshift=-1.5em]n11.west) {处理,包括:数据清洗、};
\node [anchor=west] (n13) at ([xshift=0em,yshift=-1.5em]n12.west) {子词切分、译文后处理};
\node [anchor=west] (n14) at ([xshift=0em,yshift=-1.5em]n13.west) {等};


\node [anchor=west] (n31) at ([xshift=2em,yshift=0em]n3.north east) {神经网络模型设计,包括};
\node [anchor=west] (n32) at ([xshift=0em,yshift=-1.5em]n31.west) {编码器、解码器、注意力};
\node [anchor=west] (n33) at ([xshift=0em,yshift=-1.5em]n32.west) {机制的设计};


\node [anchor=west] (n21) at ([xshift=0em,yshift=-2em]n33.south west) {在训练数据上优化模型参};
\node [anchor=west] (n22) at ([xshift=0em,yshift=-1.5em]n21.west) {数,包括训练的策略、损};
\node [anchor=west] (n23) at ([xshift=0em,yshift=-1.5em]n22.west) {失函数设计、超参数的调};
\node [anchor=west] (n24) at ([xshift=0em,yshift=-1.5em]n23.west) {整};


\node [anchor=west] (n41) at ([xshift=0em,yshift=-2em]n24.south west) {使用训练好的模型在新的};
\node [anchor=west] (n42) at ([xshift=0em,yshift=-1.5em]n41.west) {数据上进行翻译,包括解};
\node [anchor=west] (n43) at ([xshift=0em,yshift=-1.5em]n42.west) {码策略的选择、压缩、优};
\node [anchor=west] (n44) at ([xshift=0em,yshift=-1.5em]n43.west) {化等};

}

\begin{pgfonlayer}{background}
\node [rectangle,inner sep=0.2em,rounded corners=1pt,thick,draw,fill=red!5!white] [fit =  (n31)  (n32) (n33)] (box1) {};
\node [rectangle,inner sep=0.2em,rounded corners=1pt,thick,draw,fill=yellow!5!white] [fit =  (n21)  (n22) (n23) (n24) ] (box2) {};
\node [rectangle,inner sep=0.2em,rounded corners=1pt,thick,draw,fill=green!5!white] [fit = (n41) (n42) (n43) (n44) ] (box3) {};
\node [rectangle,inner sep=0.2em,rounded corners=1pt,thick,draw,fill=blue!5!white] [fit = (n11) (n12) (n13) (n14) ] (box4) {};
\end{pgfonlayer}

\draw [->,dotted,very thick,red] (n3.east) -- ([xshift=1.4em]n3.east);
\draw [->,dotted,very thick] (n2.east) -- ([xshift=1.4em]n2.east);
\draw [->,dotted,very thick,ugreen] (n4.east) -- ([xshift=1.4em]n4.east);
\draw [->,dotted,very thick,blue] (n1.west) -- ([xshift=-1.4em]n1.west);

\end{scope}
\end{tikzpicture}