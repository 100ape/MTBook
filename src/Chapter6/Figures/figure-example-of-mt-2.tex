\begin{tikzpicture}
%-------译文2
\node [pos=0.4,left,xshift=-36em,yshift=7.3em,font=\small] (original0) {译文2:};
\node [pos=0.4,left,xshift=-2em,yshift=4.5em,font=\small] (original1) {
\begin{tabular}[t]{l}
\parbox{36em}{在苏联时期,如果一个城市的人口超过一百万,它就有资格拥有自己的地铁。 规划者想要照亮日常苏联公民的生活,并把拥有数万名每日乘客的地铁看作是这样做的一个绝佳机会。 1977年,乌兹别克斯坦首都塔什干成为苏联第七个修建地铁的城市。 随着艺术的委托和设计师们的工作,乌兹别克斯坦和苏联历史的宏伟主题被赋予了生命力。 这些电台反映了不同的主题,有的有穹顶和彩砖,让人想起乌兹别克斯坦的丝绸之路清真寺,有的则用...}
\end{tabular}
};

\begin{pgfonlayer}{background}
{
\node[rectangle,draw=ublue, inner sep=0mm] [fit =(original0)(original1)] {};
}
\end{pgfonlayer}


\end{tikzpicture}