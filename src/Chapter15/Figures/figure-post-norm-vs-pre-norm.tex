%%%------------------------------------------------------------------------------------------------------------
%%% 调序模型1:基于距离的调序
\begin{center}
\begin{tikzpicture}

\begin{scope}[minimum height = 20pt]

\node [anchor=east] (x1) at (-0.5em, 0) {$\mathbi{x}_l$};
\node [anchor=west,draw,fill=red!20,inner xsep=5pt,rounded corners=2pt,thick] (F1) at ([xshift=4em]x1.east){\small{$F$}};
\node [anchor=west,circle,draw,minimum size=1em] (n1) at ([xshift=4em]F1.east) {};
\node [anchor=west,draw,fill=green!20,inner xsep=5pt,rounded corners=2pt,thick] (ln1) at ([xshift=4em]n1.east){\small{\textrm{LN}}};
\node [anchor=west] (x2) at ([xshift=4em]ln1.east) {$\mathbi{x}_{l+1}$};

\node [anchor=north] (x3) at ([yshift=-5em]x1.south) {$\mathbi{x}_l$};
\node [anchor=west,draw,fill=green!20,inner xsep=5pt,rounded corners=2pt,thick] (F2) at ([xshift=4em]x3.east){\small{\textrm{LN}}};
\node [anchor=west,draw,fill=red!20,inner xsep=5pt,rounded corners=2pt,thick] (ln2) at ([xshift=4em]F2.east){\small{$F$}};
\node [anchor=west,circle,draw,,minimum size=1em] (n2) at ([xshift=4em]ln2.east){};
\node [anchor=west] (x4) at ([xshift=4em]n2.east) {$\mathbi{x}_{l+1}$};

\draw[->, line width=1pt] ([xshift=-0.1em]x1.east)--(F1.west);
\draw[->, line width=1pt] ([xshift=-0.1em]F1.east)--(n1.west);
\draw[->, line width=1pt] (n1.east)--node[above]{$\mathbi{y}_l$}(ln1.west);
\draw[->, line width=1pt] ([xshift=-0.1em]ln1.east)--(x2.west);
\draw[->, line width=1pt] ([xshift=-0.1em]x3.east)--(F2.west);
\draw[->, line width=1pt] ([xshift=-0.1em]F2.east)--(ln2.west);
\draw[->, line width=1pt] ([xshift=0.1em]ln2.east)--node[above]{$\mathbi{y}_l$}(n2.west);
\draw[->, line width=1pt] (n2.east)--(x4.west);
\draw[->,rounded corners,line width=1pt] ([yshift=-0.2em]x1.north) -- ([yshift=1em]x1.north) -- ([yshift=1.4em]n1.north) -- (n1.north);
\draw[->,rounded corners,line width=1pt] ([yshift=-0.2em]x3.north) -- ([yshift=1em]x3.north) -- ([yshift=1.4em]n2.north) -- (n2.north);
\draw[-] (n1.west)--(n1.east);
\draw[-] (n1.north)--(n1.south);
\draw[-] (n2.west)--(n2.east);
\draw[-] (n2.north)--(n2.south);

\node [anchor=south] (k1) at ([yshift=-0.1em]x1.north){};
\node [anchor=south] (k2) at ([yshift=-0.1em]x3.north){};
\begin{pgfonlayer}{background}
\node [rectangle,inner sep=0.3em,fill=orange!10] [fit = (x1) (F1) (n1) (ln1) (x2) (k1)] (box0) {};
\node [rectangle,inner sep=0.3em,fill=blue!10] [fit = (x3) (F2) (n2) (ln2) (x4) (k2)] (box1) {};
\end{pgfonlayer}

\node [anchor=north] (c1) at (box0.south){\footnotesize {(a)Post-Norm方式的残差连接}};
\node [anchor=north] (c2) at (box1.south){\footnotesize {(b)Pre-Norm方式的残差连接}};
\end{scope}
\end{tikzpicture}
\end{center}