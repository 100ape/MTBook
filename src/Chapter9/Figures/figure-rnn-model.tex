\begin{tikzpicture}
\begin{scope}
\tikzstyle{rnnnode} = [draw,inner sep=5pt,minimum width=4em,minimum height=1.5em,fill=green!30!white,blur shadow={shadow xshift=1pt,shadow yshift=-1pt}]
\node [anchor=west,rnnnode] (node11) at (0,0) {\scriptsize{RNN Cell}};
\node [anchor=west,rnnnode] (node12) at ([xshift=2em]node11.east) {\scriptsize{RNN Cell}};
\node [anchor=west,rnnnode] (node13) at ([xshift=2em]node12.east) {\scriptsize{RNN Cell}};
\node [anchor=west,rnnnode] (node14) at ([xshift=2em]node13.east) {\scriptsize{RNN Cell}};

\node [anchor=north,rnnnode,fill=red!30!white] (e1) at ([yshift=-1.2em]node11.south) {\scriptsize{embedding}};
\node [anchor=north,rnnnode,fill=red!30!white] (e2) at ([yshift=-1.2em]node12.south) {\scriptsize{embedding}};
\node [anchor=north,rnnnode,fill=red!30!white] (e3) at ([yshift=-1.2em]node13.south) {\scriptsize{embedding}};
\node [anchor=north,rnnnode,fill=red!30!white] (e4) at ([yshift=-1.2em]node14.south) {\scriptsize{embedding}};
\node [anchor=north] (w1) at ([yshift=-1em]e1.south) {\footnotesize{亚伦}};
\node [anchor=north] (w2) at ([yshift=-1em]e2.south) {\footnotesize{任职}};
\node [anchor=north] (w3) at ([yshift=-1em]e3.south) {\footnotesize{于}};
\node [anchor=north] (w4) at ([yshift=-1em]e4.south) {\footnotesize{苹果}};

\draw [->,thick] ([yshift=0.1em]w1.north)--([yshift=-0.1em]e1.south);
\draw [->,thick] ([yshift=0.1em]w2.north)--([yshift=-0.1em]e2.south);
\draw [->,thick] ([yshift=0.1em]w3.north)--([yshift=-0.1em]e3.south);
\draw [->,thick] ([yshift=0.1em]w4.north)--([yshift=-0.1em]e4.south);

\draw [->,thick] ([yshift=0.1em]e1.north)--([yshift=-0.1em]node11.south);
\draw [->,thick] ([yshift=0.1em]e2.north)--([yshift=-0.1em]node12.south);
\draw [->,thick] ([yshift=0.1em]e3.north)--([yshift=-0.1em]node13.south);
\draw [->,thick] ([yshift=0.1em]e4.north)--([yshift=-0.1em]node14.south);

\node [anchor=south,rnnnode] (node21) at ([yshift=1.5em]node11.north) {\scriptsize{RNN Cell}};
\node [anchor=south,rnnnode] (node22) at ([yshift=1.5em]node12.north) {\scriptsize{RNN Cell}};
\node [anchor=south,rnnnode] (node23) at ([yshift=1.5em]node13.north) {\scriptsize{RNN Cell}};
\node [anchor=south,rnnnode] (node24) at ([yshift=1.5em]node14.north) {\scriptsize{RNN Cell}};

\node [anchor=south] (node31) at ([yshift=1.0em]node21.north) {\scriptsize{的表示}};
\node [anchor=south west] (node31new) at ([yshift=-0.3em]node31.north west) {\scriptsize{“亚伦”}};
\node [anchor=south] (node32) at ([yshift=1.0em]node22.north) {\scriptsize{的表示\ \ \ }};
\node [anchor=south west] (node32new) at ([yshift=-0.3em]node32.north west) {\scriptsize{“亚伦 任职”}};
\node [anchor=south] (node33) at ([yshift=1.0em]node23.north) {\scriptsize{的表示\ \ \ \ \ \ \ \ }};
\node [anchor=south west] (node33new) at ([yshift=-0.3em]node33.north west) {\scriptsize{“亚伦 任职 于”}};
\node [anchor=south] (node34) at ([yshift=1.0em]node24.north) {\scriptsize{的表示\ \ \ \ \ \ \ \ }};
\node [anchor=south west] (node34new) at ([yshift=-0.3em]node34.north west) {\scriptsize{“亚伦 任职 于 苹果”}};

\draw [->,thick] ([yshift=0.1em]node21.north)--([yshift=-0.1em]node31.south);
\draw [->,thick] ([yshift=0.1em]node22.north)--([yshift=-0.1em]node32.south);
\draw [->,thick] ([yshift=0.1em]node23.north)--([yshift=-0.1em]node33.south);
\draw [->,thick] ([yshift=0.1em]node24.north)--([yshift=-0.1em]node34.south);

\draw [->,thick] ([xshift=-1em]node21.west)--([xshift=-0.1em]node21.west);
\draw [->,thick] ([xshift=0.1em]node21.east)--([xshift=-0.1em]node22.west);
\draw [->,thick] ([xshift=0.1em]node22.east)--([xshift=-0.1em]node23.west);
\draw [->,thick] ([xshift=0.1em]node23.east)--([xshift=-0.1em]node24.west);
\draw [->,thick] ([xshift=0.1em]node24.east)--([xshift=1em]node24.east);

\draw [->,thick] ([yshift=0.1em]node11.north)--([yshift=-0.1em]node21.south);
\draw [->,thick] ([yshift=0.1em]node12.north)--([yshift=-0.1em]node22.south);
\draw [->,thick] ([yshift=0.1em]node13.north)--([yshift=-0.1em]node23.south);
\draw [->,thick] ([yshift=0.1em]node14.north)--([yshift=-0.1em]node24.south);

\draw [->,thick] ([xshift=-1em]node11.west)--([xshift=-0.1em]node11.west);
\draw [->,thick] ([xshift=0.1em]node11.east)--([xshift=-0.1em]node12.west);
\draw [->,thick] ([xshift=0.1em]node12.east)--([xshift=-0.1em]node13.west);
\draw [->,thick] ([xshift=0.1em]node13.east)--([xshift=-0.1em]node14.west);
\draw [->,thick] ([xshift=0.1em]node14.east)--([xshift=1em]node14.east);

{
\node [anchor=south] (toplabel1) at ([yshift=2em,xshift=-1.3em]node32new.north) {\footnotesize{“苹果”的表示:}};
\node [anchor=west,fill=blue!20!white,minimum width=3em] (toplabel2) at (toplabel1.east) {\footnotesize{上下文}};
}
{
\node [anchor=west,fill=red!20!white,minimum width=3em] (toplabel3) at (toplabel2.east) {\footnotesize{词}};
}

\begin{pgfonlayer}{background}
{
\node [rectangle,inner sep=2pt,draw,thick,dashed,red] [fit = (e4)] (r2) {};
\draw [->,thick,red] (r2.west) .. controls +(west:0.8) and +(south:2) .. ([xshift=1.3em]toplabel3.south);
}
{
\node [rectangle,inner sep=2pt,draw,thick,dashed,ublue,fill=white] [fit = (node33) (node33new)] (r1) {};
\draw [->,thick,ublue] ([xshift=-2em]r1.north) .. controls +(north:0.7) and +(south:0.7) .. ([xshift=-0.5em]toplabel2.south);
}
\end{pgfonlayer}

\end{scope}
\end{tikzpicture}