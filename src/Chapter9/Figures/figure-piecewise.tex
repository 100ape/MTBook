%%%------------------------------------------------------------------------------------------------------------
\begin{tikzpicture}

%% a two-layer neural network
\begin{scope}[xshift=2in]
\tikzstyle{neuronnode} = [minimum size=1.7em,circle,draw,ublue,very thick,inner sep=1pt, fill=white,align=center,drop shadow={shadow xshift=0.1em,shadow yshift=-0.1em}]
%% output illustration
\begin{scope}[xshift=2.8in,yshift=0.1in]
{
\draw [->,thick] (-2.2,0) -- (2.2,0);
\draw [->,thick] (0,0) -- (0,2);
\node [anchor=south] (heng1) at (1.95,-0.35) {\scriptsize{$x$}};
\node [anchor=south] (zong1) at (-0.2,1.6) {\scriptsize{$y$}};
\draw [-] (-0.05,1) -- (0.05,1);
\node [anchor=north,inner sep=1pt] (labelb) at (0,-0.2) {\small{(b)}};
}
{
\draw [->,thick] (-2.2,0) -- (2.2,0);
\draw [->,thick] (0,0) -- (0,2);
\draw [-,very thick,red,domain=-1.98:2,samples=100] plot (\x,{0.2 * (\x +0.4)^3 + 1.2 - 0.3 *(\x + 0.8)^2});
}
\foreach \n in {-1.9,-1.7,...,1.9}{
    \pgfmathsetmacro{\result}{0.2 * (\n + 0.1 + 0.4)^3 + 1.2 - 0.3 *(\n + 0.1 + 0.8)^2};
    \draw [-,ublue,thick] (\n,0) -- (\n, \result) -- (\n + 0.2, \result) -- (\n + 0.2, 0);
}
\end{scope}
\end{scope}

%% a two-layer neural network
\begin{scope}[xshift=0in]
\tikzstyle{neuronnode} = [minimum size=1.7em,circle,draw,ublue,very thick,inner sep=1pt, fill=white,align=center,drop shadow={shadow xshift=0.1em,shadow yshift=-0.1em}]
%% output illustration
\begin{scope}[xshift=2.8in,yshift=0.1in]
{
\draw [->,thick] (-2.2,0) -- (2.2,0);
\draw [->,thick] (0,0) -- (0,2);
\node [anchor=south] (heng1) at (1.95,-0.35) {\scriptsize{$x$}};
\node [anchor=south] (zong1) at (-0.2,1.6) {\scriptsize{$y$}};
\draw [-] (-0.05,1) -- (0.05,1);
\node [anchor=east,inner sep=1pt] (label1) at (0,1.18) {\tiny{1}};
\node [anchor=south east,inner sep=1pt] (label2) at (0,0) {\tiny{0}};
\node [anchor=north,inner sep=1pt] (labela) at (0,-0.2) {\small{(a)}};
}
{
\draw [->,thick] (-2.2,0) -- (2.2,0);
\draw [->,thick] (0,0) -- (0,2);
\draw [-,very thick,red,domain=-1.98:2,samples=100] plot (\x,{0.2 * (\x +0.4)^3 + 1.2 - 0.3 *(\x + 0.8)^2});
}
\end{scope}

\end{scope}
\end{tikzpicture}
%%%------------------------------------------------------------------------------------------------------------

