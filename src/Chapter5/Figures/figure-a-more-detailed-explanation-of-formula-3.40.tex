

%%% outline
%-------------------------------------------------------------------------




\begin{tikzpicture}

\node [anchor=west,inner sep=2pt,minimum height=2em] (eq1) at (0,0) {$f(s_u|t_v)$};
\node [anchor=west,inner sep=2pt] (eq2) at ([xshift=-2pt]eq1.east) {$=$};
\node [anchor=west,inner sep=2pt,minimum height=2em] (eq3) at ([xshift=-2pt]eq2.east) {$\lambda_{t_v}^{-1}$};
\node [anchor=west,inner sep=2pt,minimum height=3.0em] (eq4) at ([xshift=-3pt]eq3.east) {\footnotesize{$\frac{\varepsilon}{(l+1)^{m}} \prod\limits_{j=1}^{m} \sum\limits_{i=0}^{l} f(s_j|t_i)$}};
\node [anchor=west,inner sep=2pt,minimum height=3.0em] (eq5) at ([xshift=1pt]eq4.east) {\footnotesize{$\sum\limits_{j=1}^{m} \delta(s_j,s_u) \sum\limits_{i=0}^{l} \delta(t_i,t_v)$}};
\node [anchor=west,inner sep=2pt,minimum height=3.0em] (eq6) at ([xshift=1pt]eq5.east) {$\frac{f(s_u|t_v)}{\sum_{i=0}^{l}f(s_u|t_i)}$};


{
\node [anchor=west,inner sep=2pt,fill=red!20,minimum height=3.0em] (eq4) at ([xshift=-3pt]eq3.east) {\footnotesize{$\frac{\epsilon}{(l+1)^{m}} \prod\limits_{j=1}^{m} \sum\limits_{i=0}^{l} f(s_j|t_i)$}};
}
{
\node [anchor=west,inner sep=2pt,fill=blue!20,minimum height=3.0em] (eq5) at ([xshift=1pt]eq4.east) {\footnotesize{$\sum\limits_{j=1}^{m} \delta(s_j,s_u) \sum\limits_{i=0}^{l} \delta(t_i,t_v)$}};
}
{
\node [anchor=west,inner sep=2pt,fill=green!20,minimum height=3.0em] (eq6) at ([xshift=1pt]eq5.east) {$\frac{f(s_u|t_v)}{\sum_{i=0}^{l}f(s_u|t_i)}$};
}

{
\node [anchor=south west,inner sep=2pt] (label1) at (eq4.north west) {{\scriptsize{翻译概率$\textrm{P}(\mathbf{s}|\mathbf{t})$}}};
}
{
\node [anchor=south west,inner sep=2pt] (label2) at (eq5.north west) {{\scriptsize{配对的总次数}}};
\node [anchor=south west,inner sep=2pt] (label2part2) at ([yshift=-3pt]label2.north west) {{\scriptsize{$(s_u,t_v)$在句对$(\seq{s},\seq{t})$中}}};
}
{
\node [anchor=south west,inner sep=2pt] (label3) at (eq6.north west) {{\scriptsize{有的$t_i$的相对值}}};
\node [anchor=south west,inner sep=2pt] (label4) at ([yshift=-3pt]label3.north west) {{\scriptsize{$f(s_u|t_v)$对于所}}};
}

{
\node [anchor=east,rotate=90] (neweq1) at ([yshift=-0em]eq4.south) {=};
\node [anchor=north,inner sep=1pt] (neweq1full) at (neweq1.west) {\large{$\funp{P}(\seq{s}|\seq{t})$}};
}

{
\draw[decorate,thick,decoration={brace,amplitude=5pt,mirror}] ([yshift=-0.2em]eq5.south west) -- ([yshift=-0.2em]eq6.south east) node [pos=0.4,below,xshift=-0.0em,yshift=-0.3em] (expcount1) {\footnotesize{{``$t_v$翻译为$s_u$''这个事件}}};
\node [anchor=north west] (expcount2) at ([yshift=0.5em]expcount1.south west) {\footnotesize{{出现次数的期望的估计}}};
\node [anchor=north west] (expcount3) at ([yshift=0.5em]expcount2.south west) {\footnotesize{{称之为期望频次}}};
}

\end{tikzpicture}




