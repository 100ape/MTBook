%%%------------------------------------------------------------------------------------------------------------
%%%  规则抽取
\begin{center}
\begin{tikzpicture}

{\scriptsize
\begin{scope}[scale = 0.9, sibling distance=20pt, level distance=30pt]

{\footnotesize
\Tree[.\node(n1){IP};
     	[.\node(n2){NP}; [.\node(n3){PN}; \node(cw1){他}; ]]
     	[.\node(n4){VP};
     		[.\node(n5){PP};
     			[.\node(n6){P}; \node(cw2){对}; ]
     			[.\node(n7){NP};
                    [.\node(n8){NN}; \node(cw3){回答}; ]
                ]
     		]
     		[.\node(n9){VP};
     			[.\node(n10){VV}; \node(cw4){表示}; ]
     			[.\node(n11){NN}; \node(cw5){满意}; ]
     		]
     	]
     ]
}

\begin{pgfonlayer}{background}

{
\node [rectangle,fill=red!20,inner sep=0] [fit = (n11)] (n11box) {};
\node [rectangle,fill=blue!20,inner sep=0] [fit = (n4)] (n4box) {};
\node [rectangle,fill=blue!20,inner sep=0] [fit = (n1)] (n1box) {};
\node [rectangle,fill=blue!20,inner sep=0] [fit = (n2)] (n2box) {};
\node [rectangle,fill=blue!20,inner sep=0] [fit = (n3)] (n3box) {};
\node [rectangle,fill=blue!20,inner sep=0] [fit = (n5)] (n5box) {};
\node [rectangle,fill=blue!20,inner sep=0] [fit = (n6)] (n6box) {};
\node [rectangle,fill=blue!20,inner sep=0] [fit = (n7)] (n7box) {};
\node [rectangle,fill=blue!20,inner sep=0] [fit = (n8)] (n8box) {};
\node [rectangle,fill=blue!20,inner sep=0] [fit = (n9)] (n9box) {};
\node [rectangle,fill=red!20,inner sep=0] [fit = (n10)] (n10box) {};

\node [anchor=north west, minimum size=1.2em, fill=blue!20] (land1) at ([xshift=7.0em,yshift=0em]n1.north east) {};
\node [anchor=west] (land1label) at (land1.east) {\scriptsize{可信}};
\node [anchor=north west, minimum size=1.2em, fill=red!20] (land2) at ([yshift=-0.3em]land1.south west) {};
\node [anchor=west] (land2label) at (land2.east) {\scriptsize{不可信}};
}

\end{pgfonlayer}

\node[anchor=north,minimum size=18pt] (tw1) at ([yshift=-10.0em]cw1.south){he};
\node[anchor=west,minimum size=18pt] (tw2) at ([yshift=-0.1em,xshift=0.3em]tw1.east){was};
\node[anchor=west,minimum size=18pt] (tw3) at ([yshift=0.1em,xshift=0.3em]tw2.east){satisfied};
\node[anchor=west,minimum size=18pt] (tw4) at ([xshift=0.3em]tw3.east){with};
\node[anchor=west,minimum size=18pt] (tw5) at ([xshift=0.3em]tw4.east){the};
\node[anchor=west,minimum size=18pt] (tw6) at ([yshift=-0.1em,xshift=0.3em]tw5.east){answer};

\node[anchor=north](pos1) at ([xshift=-1.0em,yshift=-1.0em]tw4.south){\small{(a)标有可信节点信息的句法树}};

\draw[dashed] (cw1.south) -- ([yshift=-0.4em]tw1.north);
\draw[dashed] (cw2.south) .. controls +(south:2.0) and +(north:0.6) .. ([yshift=-0.4em]tw4.north);
\draw[dashed] (cw3.south) -- ([yshift=-0.4em]tw5.north);
\draw[dashed] (cw3.south) -- ([yshift=-0.4em]tw6.north);
\draw[dashed] (cw4.south) .. controls +(south:2.5) and +(north:0.6) .. ([yshift=-0.4em]tw3.north);
\draw[dashed] (cw5.south) .. controls +(south:2.5) and +(north:0.6) .. ([yshift=-0.4em]tw3.north);

\end{scope}

\begin{scope} [yshift = -1.87in, xshift = 2.2in]
{
\begin{scope}[level distance=20pt]
\node[scale=0.8, inner sep=0.1cm,align=center,draw] (cfrag1) at
   (0,0.25) {\Tree[.\node(sn1){NP}; [.\node(sn2){PN}; 他 ]]};
\end{scope}

\begin{scope}[level distance=20pt]
\node[scale=0.8, inner sep=0.1cm,align=center,anchor=south west,draw] (cfrag2) at
   ([xshift=1.2em]cfrag1.south east) {\Tree[.\node(sn3){P}; 对 ]};
\end{scope}

\begin{scope}[level distance=20pt]
\node[scale=0.8, inner sep=0.1cm,align=center,anchor=south west,draw] (cfrag3) at
   ([xshift=1.2em]cfrag2.south east) {\Tree[.\node(sn4){NP}; [.NN 回答 ]]};
   \end{scope}

\begin{scope}[sibling distance=15pt,level distance=20pt]
\node[scale=0.8, inner sep=0.1cm,align=center,anchor=south west,draw] (cfrag4) at
   ([xshift=1.4em]cfrag3.south east) {\Tree[.\node(sn5){VP}; [.\node(sn6){VV}; 表示 ] [.\node(sn7){NN}; 满意 ]]};
\end{scope}

\begin{scope}[sibling distance=20pt,,level distance=25pt]
\node[scale=0.8, inner sep=0.1cm,align=center,anchor=south west,draw] (cfrag5) at
   ([xshift=0.3em,yshift=2.5em]cfrag2.north west) {\Tree[.\node(sn8){PP}; [.\node(sn9){P}; ] [.\node(sn10){NP}; ]]};
\end{scope}

\begin{scope}[sibling distance=60pt]
\node[scale=0.8, inner sep=0.1cm,align=center,anchor=south west,draw] (cfrag6) at
   ([xshift=1.6em,yshift=0.8em]cfrag5.north west) {\Tree[.\node(sn11){VP}; [.\node(sn12){PP}; ] [.\node(sn13){VP}; ]]};
\end{scope}

\begin{scope}[sibling distance=80pt,level distance=18pt]
\node[scale=0.8, inner sep=0.1cm,align=center,anchor=south east,draw] (cfrag7) at
   ([xshift=-3.6em,yshift=0.8em]cfrag6.north east) {\Tree[.\node(sn14){IP}; [.\node(sn15){NP}; ] [.\node(sn16){VP}; ]]};
\end{scope}

\node[scale=0.9,anchor=north,minimum size=18pt] (tw11) at ([xshift=-0.3em,yshift=-1.2em]cfrag1.south){he};
\node[scale=0.9,anchor=west,minimum size=18pt] (tw12) at ([yshift=-0.1em,xshift=0.5em]tw11.east){was};
\node[scale=0.9,anchor=west,minimum size=18pt] (tw13) at ([yshift=0.1em,xshift=0.5em]tw12.east){satisfied};
\node[scale=0.9,anchor=west,minimum size=18pt] (tw14) at ([xshift=0.5em]tw13.east){with};
\node[scale=0.9,anchor=west,minimum size=18pt] (tw15) at ([xshift=0.5em]tw14.east){the};
\node[scale=0.9,anchor=west,minimum size=18pt] (tw16) at ([yshift=-0.1em,xshift=0.5em]tw15.east){answer};

\node[anchor=north](pos1) at ([xshift=-1.0em,yshift=-0.6em]tw14.south){\small{(b)通过边缘集合定义切割得到的句法树}};

\draw[dashed] ([xshift=-0.3em]cfrag1.south) -- ([yshift=-0.3em]tw11.north);
\draw[dashed] (cfrag2.south) -- ([yshift=-0.4em]tw14.north);
\draw[dashed] (cfrag3.south) -- ([yshift=-0.4em]tw15.north);
\draw[dashed] (cfrag3.south) -- ([yshift=-0.4em]tw16.north);
\draw[dashed] (cfrag4.south) .. controls +(south:0.6) and +(north:0.6) .. ([yshift=-0.4em]tw13.north);

\draw[*-*] ([xshift=0.0em,yshift=-0.2em]cfrag1.north) -- ([xshift=0.0em,yshift=11.3em]cfrag1.north);
\draw[*-*] ([xshift=0.1em,yshift=-0.2em]cfrag2.north) -- ([xshift=0.1em,yshift=2.9em]cfrag2.north);
\draw[*-*] ([xshift=0.1em,yshift=-0.4em]cfrag3.north) -- ([xshift=0.1em,yshift=0.9em]cfrag3.north);
\draw[*-*] ([xshift=0.0em,yshift=-0.2em]cfrag4.north) -- ([xshift=0.0em,yshift=5.7em]cfrag4.north);
\draw[*-*] ([xshift=0.1em,yshift=-0.2em]cfrag5.north) -- ([xshift=0.1em,yshift=1em]cfrag5.north);
\draw[*-*] ([xshift=0.0em,yshift=-0.2em]cfrag6.north) -- ([xshift=0.0em,yshift=1em]cfrag6.north);
}
\end{scope}
}

\end{tikzpicture}
\end{center}