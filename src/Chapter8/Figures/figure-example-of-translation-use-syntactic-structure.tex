%%%------------------------------------------------------------------------------------------------------------
%%% 基于句法的模型
\begin{center}
\begin{tikzpicture}

%% example
\begin{scope}[xshift=-0.1in,yshift=-1.5in]

{\scriptsize



\node[anchor=west] (ref) at (0,0) {{\sffamily\bfseries{人工翻译:}} {\red{After}} the school team won the Championship of the China University Basketball Association for the first time ...};

\node[anchor=north west] (hifst) at ([yshift=-0.3em]ref.south west) {{\sffamily\bfseries{机器翻译:}} \blue{In}\black{} the school team won the Chinese College Basketball League Championship for the first time ...};

{
\node[anchor=north west] (synhifst) at ([yshift=-0.2em]hifst.south west) {\sffamily\bfseries{更好?:}};

\node[anchor=west, fill=red!20, inner sep=0.3em] (synhifstpart1) at ([xshift=-0.3em]synhifst.east) {After};

\node[anchor=west, fill=blue!20, inner sep=0.25em] (synhifstpart2) at ([xshift=0.1em,yshift=-0.05em]synhifstpart1.east) {the school team won the Championship of the China University Basketball Association for the first time};

\node[anchor=west] (synhifstpart3) at ([xshift=-0.2em]synhifstpart2.east) {...};
}

\node [anchor=west] (inputlabel) at ([yshift=-0.4in]synhifst.west) {\sffamily\bfseries{输入:}};

\node [anchor=west,minimum height=12pt] (inputseg1) at (inputlabel.east) {在$_1$ };
\node [anchor=west,minimum height=12pt] (inputseg2) at ([xshift=0.2em]inputseg1.east) {学校$_2$ 球队$_3$ 首次$_4$ 夺得$_5$ 中国$_6$ 大学生$_7$ 篮球$_8$ 联赛$_9$ 冠军$_{10}$};
\node [anchor=west,minimum height=12pt] (inputseg3) at ([xshift=0.2em]inputseg2.east) {后$_{11}$};
\node [anchor=west,minimum height=12pt] (inputseg4) at ([xshift=0.2em]inputseg3.east) {,$_{12}$};
\node [anchor=west,minimum height=12pt] (inputseg5) at ([xshift=0.2em]inputseg4.east) {...};

{
\node [anchor=north,inner sep=2pt] (synlabel1) at ([yshift=-0.34in]inputseg2.south) {\scriptsize{PP}};
\node [anchor=north,inner sep=2pt] (synlabel2) at ([yshift=-0.34in]inputseg4.south) {\scriptsize{PU}};
\node [anchor=north,inner sep=2pt] (synlabel3) at ([yshift=-0.34in]inputseg5.south) {\scriptsize{VP}};
\node [anchor=north,inner sep=2pt] (synlabel4) at ([xshift=1.6in,yshift=-0.35in]synlabel1.south) {\scriptsize{VP}};

\draw [-] (inputseg1.south west) -- (inputseg3.south east) -- (synlabel1.north) -- cycle;
\draw [-] (inputseg4.south) -- (synlabel2.north);
\draw [-] (inputseg5.south) -- (synlabel3.north);
\draw [-] (synlabel1.south) -- (synlabel4.north);
\draw [-] (synlabel2.south) -- (synlabel4.north);
\draw [-] (synlabel3.south) -- (synlabel4.north);
}

{
\node [anchor=north east,align=left] (nolimitlabel) at (synlabel1.south west) {\scriptsize{短语结构树很容易捕捉}\\\scriptsize{这种介词短语结构}};
}

{
\node [anchor=west,minimum height=12pt,fill=red!20] (inputseg1) at (inputlabel.east) {在$_1$ };
\node [anchor=west,minimum height=12pt,fill=blue!20] (inputseg2) at ([xshift=0.2em]inputseg1.east) {学校$_2$ 球队$_3$ 首次$_4$ 夺得$_5$ 中国$_6$ 大学生$_7$ 篮球$_8$ 联赛$_9$ 冠军$_{10}$};
\node [anchor=west,minimum height=12pt,fill=red!20] (inputseg3) at ([xshift=0.2em]inputseg2.east) {后$_{15}$};

\path [draw,->,dashed] (inputseg1.north) .. controls +(north:0.2) and +(south:0.3) ..  ([xshift=1em]synhifstpart1.south);
\path [draw,->,dashed] (inputseg3.north) .. controls +(north:0.2) and +(south:0.6) ..  ([xshift=1em]synhifstpart1.south);
\path [draw,->,dashed] ([xshift=-0.8in]inputseg2.north) --  ([xshift=1.9in]synhifstpart2.south);
}

}

\end{scope}
%% end of example

\end{tikzpicture}
\end{center}