% !Mode:: "TeX:UTF-8"
% !TEX encoding = UTF-8 Unicode

%----------------------------------------------------------------------------------------
% 机器翻译:统计建模与深度学习方法
% Machine Translation: Statistical Modeling and Deep Learning Methods
%
% Copyright 2020
% 肖桐(xiaotong@mail.neu.edu.cn) 朱靖波 (zhujingbo@mail.neu.edu.cn)
%----------------------------------------------------------------------------------------

\renewcommand\figurename{图}

%----------------------------------------------------------------------------------------
%	PREFACE
%----------------------------------------------------------------------------------------

{\color{white} 空}
\vspace{1em}
\begin{center}
{\Huge \bfnew{前\ \ \ \ 言}}
\end{center}
\vspace{2em}

\begin{spacing}{1.18}

本书全面回顾了近三十年内机器翻译的技术发展历程,并围绕{\sffamily\bfseries 机器翻译的建模}这一主题对机器翻译的技术方法进行了全面介绍。在写作中,笔者力求用朴实的语言和简洁的实例阐述机器翻译的基本模型,同时对相关的技术前沿进行讨论。其中也会涉及大量的实践经验,包括许多机器翻译系统开发的细节。从这个角度来说,本书不单单是一本理论书籍,它还结合了机器翻译的应用,给读者提供了很多机器翻译技术落地的具体思路。

本书可以供计算机相关专业高年级本科生及研究生学习之用,也可以作为自然语言处理领域,特别是机器翻译方向相关研究人员的参考资料。此外,本书各章的主题都十分明确,内容也相对集中。因此,读者也可将每章作为某一专题的学习资料。

{\sffamily\bfseries 用最简单的方式阐述机器翻译的基本思想}是笔者所期望达到的目标。但是,书中不可避免会使用一些形式化定义和算法的抽象描述,因此,笔者尽所能通过图例进行解释(截止目前本书共287张插图)。不过,本书所包含的内容较为广泛,难免会有疏漏,望读者海涵,并指出不当之处。

本书共分为四个部分,十八章。章节的顺序参考了机器翻译技术发展的时间脉络,同时兼顾了机器翻译知识体系的内在逻辑。本书的主要内容包括:

\begin{itemize}
\vspace{0.5em}
\item 第一部分:机器翻译基础
    \begin{itemize}
    \item 第一章\ 机器翻译简介
    \item 第二章\ 统计语言建模基础
    \item 第三章\ 词法分析和语法分析基础
    \item 第四章\ 翻译质量评价
    \end{itemize}
\vspace{0.5em}
\item 第二部分:统计机器翻译
    \begin{itemize}
    \item 第五章\ 基于词的机器翻译建模
    \item 第六章\ 基于扭曲度和繁衍率的翻译模型
    \item 第七章\ 基于短语的翻译模型
    \item 第八章\ 基于句法的翻译模型
    \end{itemize}
\vspace{0.5em}
\item 第三部分:神经机器翻译
    \begin{itemize}
    \item 第九章\ 人工神经网络和神经语言建模
    \item 第十章\ 基于循环神经网络的模型
    \item 第十一章\ 基于卷积神经网络的模型
    \item 第十二章\ 基于自注意力的模型
    \end{itemize}
\vspace{0.5em}
\item 第四部分:机器翻译前沿
    \begin{itemize}
    \item 第十三章\ ???
    \item 第十四章\ ???
    \item 第十五章\ ???
    \item 第十六章\ ???
    \item 第十七章\ ???
    \item 第十八章\ ???
    \end{itemize}
\end{itemize}

\vspace{0.5em}

其中,第一部分是本书的基础知识部分,包含统计建模、语言分析、机器翻译评价等。在第一章对机器翻译的历史及现状进行介绍之后,第二章通过语言建模任务将统计建模的思想阐述出来,同时这部分内容也会作为后续机器翻译模型及方法的基础。第三章重点介绍机器翻译所涉及的词法和句法分析方法,旨在为后续相关概念的使用进行铺垫,同时进一步展示统计建模思想在相关问题上的应用。第四章相对独立,系统地介绍了机器翻译结果的评价方法,这部分内容也是机器翻译建模及系统设计所需的前置知识。

本书的第二部分主要介绍统计机器翻译的基本模型。第五章是整个机器翻译建模的基础。第六章进一步对扭曲度和产出率两个概念进行介绍,同时给出相关的翻译模型,这些模型在后续章节的内容中都有涉及。第七章和第八章分别介绍了基于短语和句法的模型。它们都是统计机器翻译的经典模型,其思想也构成了机器翻译成长过程中最精华的部分。

本书的第三部分主要介绍神经机器翻译模型,该模型也是近些年机器翻译的热点。第九章介绍了神经网络和深度学习的基础知识以保证本书知识体系的完备性。同时,第九章也介绍了基于神经网络的语言模型,其建模思想在神经机器翻译中被大量使用。第十、十一、十二章分别对三种经典的神经机器翻译模型进行介绍,以模型提出的时间为序,从最初的基于循环网络的模型,到最新的Transformer模型均有涉及。其中也会对编码器-解码器框架、注意力机制等经典方法和技术进行介绍。

本书的第四部分会进一步对机器翻译的前沿技术进行讨论,以神经机器翻译为主。该部分目前正在写作中,很快就会与读者见面。

%-------------------------------------------
\begin{figure}[htp]
\centering
\centering
% !Mode:: "TeX:UTF-8"
% !TEX encoding = UTF-8 Unicode

\begin{tikzpicture}

\tikzstyle{partnode} =[font=\scriptsize,minimum height=2.0em,minimum width=15em,draw,thick,fill=white,drop shadow]
\tikzstyle{secnode} =[font=\footnotesize,minimum height=1.6em,minimum width=14em,align=flush left]

\begin{scope}

% part 1
\node [partnode,anchor=south,blue,minimum height=9.0em,minimum width=22.7em,fill=white] (part1) at ([yshift=-0.5em]0,0) {};
\node [anchor=north] (part1label) at ([yshift=-0.3em]part1.north) {\sffamily\bfseries{机器翻译基础}};
\node [anchor=north west,draw=blue,thick,fill=white,rounded corners] (part1title) at ([xshift=-0.3em,yshift=0.3em]part1.north west) {{\color{blue} {\sffamily\bfseries 第一部分}}};
\node [secnode,anchor=south,fill=ugreen!20,minimum width=21.6em,align=center] (sec01) at (0,0) {第一章\hspace{1em} 机器翻译简介};
\node [secnode,anchor=south west,fill=blue!20] (sec02) at ([yshift=0.8em]sec01.north west) {第二章\hspace{1em} 统计语言建模基础\hspace{3em}};
\node [secnode,anchor=south west,fill=blue!20] (sec03) at ([yshift=0.8em]sec02.north west) {第三章\hspace{1em} 词法分析和语法分析基础};
\node [secnode,anchor=north west,fill=blue!20,minimum width=7em,minimum height=4.1em,align=center] (sec04) at ([xshift=0.6em]sec03.north east) {第四章\\ 翻译质量评价};
\draw [->,very thick] ([yshift=-0.7em]sec02.south) -- ([yshift=-0.1em]sec02.south);
\draw [->,very thick] ([yshift=-0.7em]sec03.south) -- ([yshift=-0.1em]sec03.south);
\draw [->,very thick] ([yshift=-0.7em]sec04.south) -- ([yshift=-0.1em]sec04.south);

% part 2
\node [partnode,anchor=south,orange,minimum height=11.5em,minimum width=18.1em,fill=white] (part2) at ([yshift=3em]part1.north west) {};
\node [anchor=north] (part2label) at ([yshift=-0.3em]part2.north) {\sffamily\bfseries{统计机器翻译}};
\node [anchor=north west,draw=orange,thick,fill=white,rounded corners] (part2title) at ([xshift=-0.3em,yshift=0.3em]part2.north west) {{\color{orange} {\sffamily\bfseries 第二部分}}};
\node [secnode,anchor=south,fill=orange!20,minimum width=17em,align=left] (sec04) at ([yshift=0.5em]part2.south) {第五章\hspace{1em} 基于词的机器翻译建模 \hspace{2.35em}};
\node [secnode,anchor=south,fill=orange!20,minimum width=17em,align=center] (sec05) at ([yshift=0.8em]sec04.north) {\hspace{1.0em}第六章\hspace{1em} 基于扭曲度和繁衍率的模型\hspace{1.6em}};
\node [secnode,anchor=south,fill=orange!20,minimum width=17em,align=center] (sec06) at ([yshift=0.8em]sec05.north) {第七章\hspace{1em} 基于短语的模型 \hspace{5.35em}};
\node [secnode,anchor=south,fill=orange!20,minimum width=17em,align=center] (sec07) at ([yshift=0.8em]sec06.north) {第八章\hspace{1em} 基于句法的模型 \hspace{5.35em}};
\draw [->,very thick] ([yshift=-0.7em]sec05.south) -- ([yshift=-0.1em]sec05.south);
\draw [->,very thick] ([yshift=-0.7em]sec06.south) -- ([yshift=-0.1em]sec06.south);
\draw [->,very thick] ([yshift=-0.7em]sec07.south) -- ([yshift=-0.1em]sec07.south);

% part 3
\node [partnode,anchor=south,red,minimum height=9.5em,minimum width=22.7em,fill=white] (part3) at ([yshift=3em,xshift=2.5em]part2.north east) {};
\node [anchor=north] (part3label) at ([yshift=-0.3em]part3.north) {\sffamily\bfseries{神经机器翻译}};
\node [anchor=north west,draw=red,thick,fill=white,rounded corners] (part3title) at ([xshift=-0.3em,yshift=0.3em]part3.north west) {{\color{red} {\sffamily\bfseries 第三部分}}};
\node [secnode,anchor=south,fill=magenta!20,minimum width=21.6em,align=center] (sec09) at ([yshift=0.5em]part3.south) {第九章\hspace{1em} 人工神经网络和神经语言建模};
\node [secnode,anchor=south west,fill=red!20,minimum width=6.6em,minimum height=4.5em,align=center] (sec10) at ([yshift=0.8em]sec09.north west) {第十章\\ 基于循环神经 \\ 网络的模型};
\node [secnode,anchor=south west,fill=red!20,minimum width=6.6em,minimum height=4.5em,align=center] (sec11) at ([xshift=0.8em]sec10.south east) {第十一章\\ 基于卷积神经 \\ 网络的模型};
\node [secnode,anchor=south west,fill=red!20,minimum width=6.6em,minimum height=4.5em,align=center] (sec12) at ([xshift=0.8em]sec11.south east) {第十二章\\ 基于自注意力 \\ 的模型};
\draw [->,very thick] ([yshift=-0.7em]sec10.south) -- ([yshift=-0.1em]sec10.south);
\draw [->,very thick] ([yshift=-0.7em]sec11.south) -- ([yshift=-0.1em]sec11.south);
\draw [->,very thick] ([yshift=-0.7em]sec12.south) -- ([yshift=-0.1em]sec12.south);


% part 4
\node [partnode,anchor=south,ugreen,minimum height=12.0em,minimum width=29.7em,fill=white] (part4) at ([yshift=3em,xshift=6em]part3.north west) {};
\node [anchor=north] (part4label) at ([yshift=-0.3em]part4.north) {\sffamily\bfseries{机器翻译前沿}};
\node [anchor=north west,draw=ugreen,thick,fill=white,rounded corners] (part4title) at ([xshift=-0.3em,yshift=0.3em]part4.north west) {{\color{ugreen} {\sffamily\bfseries 第四部分}}};
\node [secnode,anchor=south west,fill=cyan!20,minimum width=14.0em,align=center] (sec13) at ([yshift=0.5em,xshift=0.5em]part4.south west) {第十三章\hspace{1em} 神经机器翻译模型训练};
\node [secnode,anchor=west,fill=cyan!20,minimum width=14.0em,align=center] (sec14) at ([xshift=0.6em]sec13.east) {第十四章\hspace{1em} 神经机器翻译模型推断};
\node [secnode,anchor=south west,fill=green!30,minimum width=9em,minimum height=4.5em,align=center] (sec15) at ([yshift=0.8em]sec13.north west) {第十五章\\ 神经机器翻译 \\ 结构优化};
\node [secnode,anchor=south west,fill=green!30,minimum width=9em,minimum height=4.5em,align=center] (sec16) at ([xshift=0.8em]sec15.south east) {第十六章\\ 低资源 \\ 神经机器翻译};
\node [secnode,anchor=south west,fill=green!30,minimum width=9em,minimum height=4.5em,align=center] (sec17) at ([xshift=0.8em]sec16.south east) {第十七章\\ 多模态、多层次 \\ 机器翻译};
\node [secnode,anchor=south west,fill=amber!25,minimum width=28.7em,align=center] (sec18) at ([yshift=0.8em]sec15.north west) {第十八章\hspace{1em} 机器翻译应用技术};
\node [rectangle,draw,dotted,thick,inner sep=0.1em,fill opacity=1] [fit = (sec13) (sec14)] (nmtbasebox) {};
\draw [->,very thick] ([yshift=-0.7em]sec15.south) -- ([yshift=-0.1em]sec15.south);
\draw [->,very thick] ([yshift=-0.7em]sec16.south) -- ([yshift=-0.1em]sec16.south);
\draw [->,very thick] ([yshift=-0.7em]sec17.south) -- ([yshift=-0.1em]sec17.south);
\draw [<-,very thick] ([yshift=0.7em]sec15.north) -- ([yshift=0.1em]sec15.north);
\draw [<-,very thick] ([yshift=0.7em]sec16.north) -- ([yshift=0.1em]sec16.north);
\draw [<-,very thick] ([yshift=0.7em]sec17.north) -- ([yshift=0.1em]sec17.north);
\draw [->,very thick,dotted] ([yshift=-0.7em,xshift=0.4em]sec15.south east) -- ([yshift=0.7em,xshift=0.4em]sec15.north east);
\draw [->,very thick,dotted] ([yshift=-0.7em,xshift=0.4em]sec16.south east) -- ([yshift=0.7em,xshift=0.4em]sec16.north east);

% lines and arrows
\draw [->,line width=0.2em] ([xshift=-0.1em]part1.west) .. controls +(west:5em) and +(south:3em) .. ([yshift=-0.1em,xshift=-5em]part2.south);
\draw [->,line width=0.2em] ([xshift=-2em,yshift=0.1em]part1.north east) -- ([xshift=-2em,yshift=17.3em]part1.north east);
\draw [->,line width=0.2em] ([xshift=0.3em,yshift=-2em]part2.east) .. controls +(east:6em) and +(south:4em) .. ([yshift=-0.3em,xshift=4em]part3.south);
\draw [->,line width=0.2em] ([xshift=-5em,yshift=0.5em]part2.north) -- ([xshift=-5em,yshift=15.3em]part2.north);
\draw [->,line width=0.2em] ([xshift=4em,yshift=0.1em]part3.north) -- ([xshift=4em,yshift=2.7em]part3.north);

\end{scope}

\end{tikzpicture}

\end{figure}
%-------------------------------------------

\end{spacing}








